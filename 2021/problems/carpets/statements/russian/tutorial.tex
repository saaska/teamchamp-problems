\begin{tutorial}{Два ковра}

\textit{1 способ.} Пересечение двух прямоугольников также будет прямоугольником.  Координаты левого нижнего угла определяются по формулам:
$$x = max(x_1, x_2); \quad y = max(y_1, y_2),$$
длины сторон~--- по формулам:
\begin{center}
$w = min(x_1 + w_1, x_2 + w_2) - x$;\\
$h = min(y_1 + h_1, y_2 + h_2) - y$.
\end{center}
Если заданные прямоугольники не пересекаются, то одна из них или обе будут отрицательны.
Ясно, что видимая  площадь будет равна:
\begin{center}
$s = w_1 \cdot h_1 - w \cdot h$, если $h>0$ и $w>0$, а иначе $s = w_1 \cdot h_1$.
\end{center}


\textit{2 способ.} Проведём по сторонам прямоугольников линии $x[i]$ и $y[i]$, $i = 0,1,2,3$. Сортируем полученные массивы по возрастанию. Получим клетчатую область, содержащую заданные прямоугольники. Осталось проверить, принадлежат ли точки середин полученных клеток заданным прямоугольникам. При этом суммируем площади, клеток принадлежащих только первому многоугольнику. Эта сумма и будет ответом.


\end{tutorial}
