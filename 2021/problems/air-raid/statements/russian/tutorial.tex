\begin{tutorial}{Бомбардировка}

Для решения этой задачи можно сделать перебор по строкам.

Рассмотрим всевозможные комбинации бомбардировок по строкам (их всего $2^n$ случаев). В каждом случае, после бомбардировок по строкам, возможно останутся какие-то вражеские базы. На каждом столбце, на котором остались базы, нужно совершить бомбардировки. Т.е. для получения количества полётов в каждом случае к количеству бомбардировок по строкам нужно прибавить количество столбцов с нетронутыми базами.

Из всех этих случаев нужно выбрать тот, в котором совершено наименьшее количество полётов.

Для реализации этого решения удобно использовать битовые операции.

\end{tutorial}
