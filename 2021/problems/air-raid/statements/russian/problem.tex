\begin{problem}{Бомбардировка}{стандартный ввод}{стандартный вывод}{2 секунды}{256 мегабайт}

Разведчики нашли военную территорию врага. Вражескую территорию можно представить как клеточное поле размера $n \times m$. Некоторые клетки заняты вражескими базами, причем одна база занимает ровно одну клетку.

Военные самолёты под вашим командованием всегда совершают вылеты строго вдоль строк или строго вдоль столбцов и могут за один полет разбомбить все базы, находящиеся в~одной строке или в~одном столбце. Вам нужно найти минимальное количество полётов, которые нужно совершить для бомбардировки всех вражеских баз. 

\InputFile
В первой строке даны два целых числа $n$ и $m$~--- размеры вражеского поля $(1 \leq n,m \leq 16)$.

В следующих $n$ строках находится по $m$ символов~--- описание поля. Описание состоит из символов <<$.$>> и <<$\#$>>.  Если $j$-й символ в $i$-й строке равен <<$\#$>>, то клетка $(i,j)$ занята вражеской базой, иначе эта клетка пуста.

\OutputFile
Выведите одно неотрицательное число~--- ответ на задачу.

\Examples

\begin{example}
\exmpfile{example.01}{example.01.a}%
\exmpfile{example.02}{example.02.a}%
\exmpfile{example.03}{example.03.a}%
\end{example}

\end{problem}

