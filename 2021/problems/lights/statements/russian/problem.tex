\begin{problem}{Люцерны}{стандартный ввод}{стандартный вывод}{1 секунда}{256 мегабайт}

Во время очередных раскопок археолог Октавий и его друг Аль-Хорезми внезапно провалились под землю и оказались в очень странном и загадочном месте. Там было очень темно и холодно, а рядом на большом камне лежали необычные древние фигурки. Октавий взял в руки одну из них, чтобы рассмотреть. К его удивлению, сразу же зажегся свет в некоторых люцернах, которые были расположены в~один ряд на большой каменной двери. Положил обратно на камень~--- те же люцерны погасли. Люцерна представляет собой нечто, похожее на лампочку, но выполненную очевидно, по очень древним и неизвестным современным людям технологиям. Поднимая и опуская другие фигурки, Октавий и Аль-Хорезми заметили, что в~зависимости от фигурки загорается и гаснет всегда один и тот же набор люцерн (для каждой фигурки он свой).

Друзья сразу поняли, что эта дверь является выходом из этого мрачного места. Но как ее открыть? Обследовав люцерны, они обнаружили, что некоторые из них помечены, и друзья решили, что нужно зажечь свет именно в них, а остальные должны остаться не зажженными. Октавий начал манипулировать фигурками наугад в надежде, что однажды ему повезет, и он сможет открыть эту дверь. Но, увы... Прошло уже немало времени, а у Октавия все никак не получалось найти подходящую комбинацию. Вдруг Аль-Хорезми понял, что есть закономерность между фигурками и люцернами:
\begin{itemize}
\item когда все фигурки лежат на камне~--- не горит ни одна люцерна;
\item если ранее какие-то люцерны уже горели, то при взятии в руки еще одной фигурки, те люцерны, которые должны были загореться, работали иначе. Если ранее люцерна была погашена, то она зажигалась, а зажженная люцерна гасла.
\end{itemize}
К счастью, у Аль-Хорезми с собой оказался ноутбук, он смог написать программу, которая решила эту головоломку, и друзья смогли выйти из ловушки. Попробуйте и вы ее решить!

\InputFile
В первой строке записаны два натуральных числа $N$ и $M$~--- количество люцерн на двери и количество фигурок ($N \le 64$, $M \le 20$).

В следующей строке записаны через пробел $N$ единиц и нулей, где единицы указывают на помеченные люцерны.

В следующих $M$ строках описано, какие люцерны зажигаются и гаснут для каждой фигурки. А именно, в $i$-й строке записано $N$ единиц и нулей для $i$-й фигурки: комбинация единиц дает набор зажигаемых и гаснущих люцерн.

\OutputFile
В первой строке вывести количество фигурок, которые необходимо поднять, чтобы открыть дверь. В~следующей строке через пробел перечислить номера фигурок. Если решений несколько, то вывести любое из них. Гарантируется, что есть хотя бы одно решение.

\Example

\begin{example}
\exmpfile{example.01}{example.01.a}%
\end{example}

\Note
В примере нужно использовать фигурки с номерами 1, 3 и 4. 

Первая фигурка зажжет люцерны с номерами 1 и 3. Третья обычно зажигает люцерны 1, 2 и 4, но если следовать условию, что ранее зажженные гаснут, то теперь вместе они зажгут люцерны 2, 3 и 4.

Четвертая фигурка зажигает только четвертую люцерну. Таким образом, останутся только вторая и третья люцерна, что совпадает с помеченными люцернами на двери.

\end{problem}

