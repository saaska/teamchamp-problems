\begin{tutorial}{TWICE}

Задача решается очень просто. 

Находим сумму $A$ и $B$ и умножаем на два. После этого находим разность $C$ и $D$ и возводим результат в квадрат.

Сравниваем оба результата. Если удвоенная сумма больше, печатаем $A$ и $B$, а если разность больше, выводим $C$ и $D$. Если же два результата равны, то выводим $A$, $B$, $C$, $D$.

Пример кода на языке программирования Python:
\begin{verbatim}
a,b,c,d = list(map(int,input().split()))

if (a+b)*2>(c-d)**2:
    print(a,b)
elif (a+b)*2<(c-d)**2:
    print(c,d)
else:
    print(a,b,c,d)
\end{verbatim}

\end{tutorial}
