\begin{problem}{Самый опасный индивид}{стандартный ввод}{стандартный вывод}{3 секунды}{256 мегабайт}

В стране Z распространилась неведомая очень заразная инфекционная болезнь. Специалисты Центра по контролю и профилактике инфекционных заболеваний (ЦКЗ) хотят в~первую очередь найти и изолировать наиболее опасного индивида, а именно того, который имеет наибольший потенциал распространить инфекционное заболевание в~обществе. Этому может помочь очень хорошая системная работа, которую ведет ЦКЗ вот уже на протяжении нескольких лет. За это время ЦКЗ накопил и постоянно обновляет данные обо всех близких связях между людьми. Считается, что два человека имеют близкую связь, если они ежедневно встречаются друг с~другом. При этом через близкие связи тех, с кем у человека имеется близкая связь, каждый в~этой стране связан с~каждым, но ни одна цепочка связей не приводит к нему обратно. 

Помогите ЦКЗ найти самого опасного индивида страны Z, если известно, что в~ней проживает $n$ человек, а показателем опасности индивида $i$ является величина: 

$$ \displaystyle\frac{n-1}{\sum_{j \neq i} l(i,j)}  ,$$
где $l(i,j)$~-- длина кратчайшей цепочки близких связей между $i$ и $j$ индивидами.
 
Чем больше данный показатель, тем опаснее инидивид.  

\InputFile
Первая строка содержит целое число $n$ ($1 \leq n \leq 600 000$)~--- население страны Z. Вторая строка содержит информацию о близких связях жителей страны, заданную следующим образом: записано $n-1$ целых чисел (в~порядке от первого до ($n-1$)-го) и $i$-е число соответствует номеру индивида, с которым индивид $i$ имеет близкую связь. 

Обратите внимание, что ЦКЗ пронумеровал всех людей страны номерами от $0$ до $n-1$.  

\OutputFile
Выведите одно целое число~--- номер самого опасного индивида.  Если самых опасных индивидов несколько, то выведите наименьший номер.

\Examples

\begin{example}
\exmpfile{example.01}{example.01.a}%
\exmpfile{example.02}{example.02.a}%
\end{example}

\end{problem}

