\begin{problem}{Король обезьян}{стандартный ввод}{стандартный вывод}{1 секунда}{256 мегабайт}

2050 год... Молодой якутский натуралист Хатан в составе группы ученых из Земли изучает животный мир четвертой планеты звездной системы Альфы Центавра. Более всего земным ученым интересны обезьяноподобные животные~--- мартальфы, т.к. их поведение обнаруживает зачатки разума. 

Мартальфы построили единую иерархическую модель отношений, на вершине которой стоит альфа-особь. Эта особь может повелевать всеми мартальфами. И поведение этой особи представляет особый интерес для ученых. Но проблема в том, что внешне он ничем не выделяется от остальных особей, а изучить его надо обязательно. Поэтому перед Хатаном поставили задачу определить альфа-особь мартальфов.

После длительного изучения сообщества мартальфов он пришел к выводу, что показателем их положения в сообществе являются круглые пятна на теле животного. Но на положение влияет не количество и не цвет пятен, а размер самого большего пятна. У кого самое большое круглое пятно, тот и является альфа-особью! При этом размеры пятен на теле животных не повторяются, поэтому альфа-особь всегда определяется однозначно.

Хатан сфотографировал всех мартальфов и определил радиусы их круглых пятен. Помогите Хатану, напишите программу, которая сможет определить кто же является альфа-особью!

\InputFile
На вход программе дается последовательность натуральных чисел. Первая строка содержит число $N$~--- количество особей в сообществе мартальфов. Далее в $N$ строках заданы описания особей сообщества. В каждой строке первое число $M$ определяет количество пятен у особи, а далее $M$ натуральных чисел~--- радиусы пятен животного ($1<N \leq 10^4, 1<M \leq 100, 0< a_{ij} \leq 10^6$). 

\OutputFile
Ваша программа должна вывести одно число~--- порядковый номер альфа-особи (нумерация начинается с 1).

\Example

\begin{example}
\exmpfile{example.01}{example.01.a}%
\end{example}

\end{problem}

