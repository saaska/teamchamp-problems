\begin{tutorial}{Графство графа}

Решение задачи состоит из реализации трех основных пунктов:
\begin{enumerate}
\item Поиск в ширину
\item Динамика на графах
\item Длинная арифметика
\end{enumerate}

Первый пункт: мы реализуем почти стандартный поиск в ширину. Добавляем в очередь изначально вершину под номером 1 и все вершины, в которые нет путей. Это все вершины, которые не встречаются на втором месте во всех $n$ строках с ребрами. Последующие вершины мы добавляем в очередь тогда и только тогда, когда все вершины, из которых есть путь в эту вершину, уже полностью обработаны. 

Второй пункт: реализуем стандартную динамику, где количество путей до очередной вершины это сумма всех путей из всех вершин, где есть прямой путь в эту вершину. Заметьте, в самом начале только у вершины с номером 1 значение равно 1, в других вершинах значение динамики равно 0.

Третий пункт: реализация длинной арифметики, чтобы найти вершину до которой максимальное количество путей.

Объединив все все эти пункты, мы получаем решение этой задачи.

\end{tutorial}
