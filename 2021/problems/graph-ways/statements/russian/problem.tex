\begin{problem}{Графство графа}{стандартный ввод}{стандартный вывод}{1 секунда}{256 мегабайт}

Великий граф Григорий решил отдохнуть и съездить на отдых в один из маленьких милых городов его графства. 

Так как Григорий~--- очень успешный граф, у него очень много врагов, которые хотят поймать его на пути в город отдыха. Его друг и аналитик Айтал подсказал, что самый безопасный город для отдыха~--- город, до которого \textbf{наибольшее} количество различных путей.

Немного о графстве. Город проживания графа~--- город под номером 1, и все другие города тоже пронумерованы. Дороги, соединяющие города, однонаправленные. Известно, что любой путь из города в город можно однозначно задать последовательностью городов, через которые он проходит, а начальный и конечный города всегда различны. Во всем графстве  существует ограниченное количество путей, то есть графство не имеет циклов.

Помогите Григорию выбрать нужный ему город! Если таких городов несколько, выведите тот, у которого наименьший номер.

\InputFile
В первой строке заданы два натуральных числа $N$ и $M$ ($1 \leq N, M \leq 10^5$)~--- количество городов и дорог в графстве соответственно.

В следующих $M$ строках записаны по два числа $a$ и $b$, описывающих одностороннюю дорогу из города $a$ в город $b$.

\OutputFile
Выведите одно число $X$~--- номер лучшего города для отдыха Григория.

\Examples

\begin{example}
\exmpfile{example.01}{example.01.a}%
\exmpfile{example.02}{example.02.a}%
\end{example}

\end{problem}

