\begin{tutorial}{Странные числа планеты Фывапро}

Реализуем алгоритм перевода целого числа из 10-й системы счисления в $P$-ичную методом "от младших разрядов к старшим" и параллельно находим значение максимальной цифры и подсчитываем количество ее вхождений. Поскольку данный алгоритм нужно будет выполнить дважды (для каждого числа), имеет смысл оформить его в виде подпрограммы.

Пример программы на языке Паскаль:
\begin{verbatim}
var a,b,p: integer;
    m1,m2,c1,c2: integer;
    
procedure count_max(x: integer; var m,c : integer);
var r: integer;
begin
  m:=0; c:=0;
  while x>0 do
    begin
      r:=x mod p;
      x:=x div p;
      if r>m then begin m:=r; c:=1 end
      else
        if r=m then c:=c+1;
    end;
end;

begin
  readln(a,b);
  readln(p);

  count_max(a,m1,c1);
  count_max(b,m2,c2);
  
  if m1>m2 then writeln(1)
  else
    if m2>m1 then writeln(2)
    else 
      if c1>c2 then writeln(1)
      else
        if c2>c1 then writeln(2)
        else writeln(0);
end.
\end{verbatim}

\end{tutorial}
