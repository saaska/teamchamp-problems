\begin{problem}{Странные числа планеты Фывапро}{стандартный ввод}{стандартный вывод}{1 секунда}{256 мегабайт}

На планете Фывапро у жителей на руках и ногах по $P$ пальцев, и, соответственно, для записи чисел они используют $P$ различных упорядоченных по величине знаков. Однако вот что странно: из двух чисел б\'{о}льшим они считают то число, у которого в записи числа чаще встречается максимальный по величине знак. Конечно, если у чисел разные максимальные по величине знаки, то больше то число, у которого максимальный знак больше. 

Узнав об этом, молодые программисты Айсен и Валера развеселились и шутя задались вопросом: кто из них на планете Фывапро имел бы б\'{о}льшую зарплату? Для получения  записей размеров своей зарплаты в знаках планеты Фывапро они условились перевести их в~$P$-ичную систему счисления и сопоставить цифры используемым <<фывапровским>> знакам. 

Конечно, им не стоило никакого труда узнать ответ на свой вопрос, а вот вам это под силу?  

\InputFile
В первой строке записаны два натуральных числа $A$ и $B$ ($1 \le A, B \le 10^9$) ---  соответственно, суммы зарплат Айсена и Валеры за последний месяц. Во второй строке --- одно натуральное число $P$ ($2 \le P \le 16$).

\OutputFile
Вывести 1, если бы Айсен зарабатывал больше, 2, если бы это был Валера и 0, если бы их зарплаты были бы равны. 

\Examples

\begin{example}
\exmpfile{example.01}{example.01.a}%
\exmpfile{example.02}{example.02.a}%
\exmpfile{example.03}{example.03.a}%
\end{example}

\Explanation
В третьем примере в 8-ой системе счисления заданные числа $A$ и $B$ будут записаны как 127710 и 130674 соответственно. В первой записи две семерки, а во второй~--- одна. Поэтому зарплата Айсена на планете Фывапро была бы больше.

\end{problem}

