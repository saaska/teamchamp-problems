\begin{tutorial}{Путь Флипа}

Данная задача сочетает в себе стандартное двумерное динамическое программирование и элементы вычислительной геометрии.

Целевая функция будет зависеть от двух параметров: номера границы $i$ и номера островка безопасности $j$ и иметь своим значением минимальное расстояние пути кузнечика из начала координат в~точку с~координатами $(i, j)$.

Кроме этого, для восстановления маршрута минимальной длины понадобится вторая целевая функция от тех же параметров, имеющая своим значением номера точек на предыдущей границе, прыжок из которых в~точку $(i, j)$ обеспечивает минимальный путь.
 
Чтобы вычислить $j$-е координаты точек, в которых кузнечик приземляется на внешних границах дороги со стороны своего дома и со стороны дома пчелки Майи, воспользовавшись подобием треугольников, получим соответственно, формулы:
$$ 
x[0][j]=a*x[1][j]/(a+h), x[n][jmin]=x[n-1][jmin]-(x[n-1][jmin]-m)*h/(b+h).
$$

Решение задачи выполняется в два этапа: вычисление значений целевых функций от 1-й границы между полосами автострады до последней, затем определяется номер точки, на предпоследней границе, через которую проходит минимальный путь кузнечика и восстанавливается его маршрут с использованием второй целевой функции в обратном порядке. 


\end{tutorial}
