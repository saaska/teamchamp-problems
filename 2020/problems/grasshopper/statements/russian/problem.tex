\begin{problem}{Путь Флипа}{стандартный ввод}{стандартный вывод}{1 секунда}{256 мегабайт}

Кузнечик Флип и пчелка Майя~--- лучшие друзья. К великому огорчению Флипа в~этом году их дома оказались разделены многополосной автострадой. Какое дело людям до дружбы каких-то там насекомых! За все время строительства только Майя прилетала к~Флипу, но сегодня у нее день рождения, и пчелка принимает гостей у себя дома. 

К счастью, на границах полос автострады довольно много островков безопасности, и кузнечик рассчитывает перебраться через автостраду, перепрыгивая в~целях безопасности ровно через одну полосу от островка к~островку. Непонятно, чем руководствовались те, кто делал разметку, но количество островков на границах разное, да и нарисовали их безо всякой логики~--- то близко друг от друга, то далеко.

Помогите Флипу перебраться через автостраду~--- подскажите, на какие островки нужно прыгать, чтобы по длине весь путь кузнечика был минимальным. Известно, что дом Флипа находится на координатной плоскости в~точке (0, 0), границы автострады параллельны оси $x$, первый прыжок кузнечик делает на внешнюю границу автострады со стороны своего дома, а последний~--- с~внешней границы автострады со стороны дома Майи. 

\InputFile
В первой строке входных данных записаны три целых числа: $a$ и $b$~--- соответственно, кратчайшие расстояния от домов Флипа и Майи до автострады ($1 \le a, b \le 100$), $m$~--- координата дома Майи по оси $x$ ($-1000 \le m \le 1000$). 

Во второй строке заданы два целых числа $n$ ($1 \le n \le 100$)~--- количество полос автострады и $h$~--- ширина каждой полосы ($1 \le h \le 10$). В~следующих $n-1$ строках задано $k_i+1$ чисел: первое натуральное число $k_i$~--- количество островков безопасности на $i$-й границе полосы ($1 \le k_i \le 20$), остальные числа~--- координаты островков по оси $x$, все координаты целые и не превышают по модулю 1000. 



\OutputFile
Выведите построчно $x$-координаты точек приземлений кузнечика по минимальному пути с~точностью не хуже $10^{-4}$. 

Если вариантов ответа более одного, выведите любой из них. 

\Example

\begin{example}
\exmpfile{example.01}{example.01.a}%
\end{example}

\Note
$y$-координата дома Майи больше 0.

\end{problem}

