\begin{problem}{Призеры универсиады}{стандартный ввод}{стандартный вывод}{1 секунда}{256 мегабайт}

Соревнования по тяжёлой атлетике в Российской Федерации проводятся в~следующих видах программы~--- рывок, толчок и двоеборье (рывок и толчок). Факторами, учитываемыми для классификации спортсменов в сумме двоеборья являются:

1. Лучший результат: наибольший результат даёт первое место; если у спортсменов они идентичные, тогда учитываются ещё:

2. Лучший результат в~толчке: наименьший результат даёт первое место; если они идентичные, тогда учитываются ещё:

3. Номер попытки, в~которой показан лучший результат в толчке: наименьший номер попытки даёт первое место; если они идентичные, тогда учитываются ещё:

4. Предыдущая попытка (попытки): наименьший вес в попытке даёт первое место; если они идентичные, тогда учитывается ещё:

5. Номер жеребьёвки: меньший номер~--- первый.

У юного программиста Васи старший брат Петя~--- участник соревнований Всероссийской универсиады в~этом виде спорта. Он еще до объявления итогов прислал Васе протокол соревнований своей весовой категории и попросил определить тройку призеров. Вася решил составить программу, но правила соревнований оказались слишком запутанными для него. А вы смогли бы ему помочь?

\InputFile
Васе удалось немного обработать рабочий протокол, и теперь первая строка входных данных содержит натуральное число $N$~--- количество участников соревнований ($1 \le N \le 1000$). Далее следует $N$ строк, каждая из которых представляет результаты выступления отдельного спортсмена в следующем формате:
$$T\ University\ Name\ Surname\ J_1\ J_2\ J_3\ P_1\ P_2\ P_3$$
Здесь:
\begin{itemize}
\item $T$~--- номер жеребьёвки, натуральное число ($1 \le T \le N$);
\item $University, Name, Surname$~--- соответственно, вуз, за который выступает спортсмен, его имя и фамилия~--- строки из символов латинского алфавита, каждая из которых не превышает по длине 80 символов;
\item $J_1\ J_2\ J_3$~--- результаты трех попыток в~рывке в~килограммах, неотрицательные целые числа ($0 \le J_1, J_2, J_3 \le 130$), 0 означает, что попытка не увенчалась успехом и заявленный вес не был взят;
\item $P_1\ P_2\ P_3$~--- аналогично, результаты трех попыток в~толчке.
\end{itemize}

Все параметры в строке разделены ровно одним пробелом, ведущие и хвостовые пробелы отсутствуют.

\OutputFile
Построчно выведите тройку победителей (1, 2, 3 место) согласно следующему формату (между словами~--- ровно один пробел):
$$University\ Name\ Surname$$
Спортсменов без зачетных попыток хотя бы в одном виде следует проигнорировать.  Если победителей 1 или 2, следует вывести только их. Если невозможно выявить ни одного победителя, то следует вывести строку <<\texttt{No results}>>.

\Examples

\exmpwidinf=0.56\thelinewidth
\exmpwidouf=0.28\thelinewidth
\small
\begin{example}
\exmpfile{example.01}{example.01.a}%
\end{example}
\begin{example}
\exmpfile{example.02}{example.02.a}%
\exmpfile{example.03}{example.03.a}%
\end{example}
\normalsize

\end{problem}

