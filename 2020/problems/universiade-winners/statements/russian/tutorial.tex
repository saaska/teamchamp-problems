\begin{tutorial}{Призеры универсиады}

Ограничения задачи позволяют отсортировать весь список спортсменов в порядке <<от лучших к~худшим>> и определить тройку призеров. Хотя, как известно, можно реализовать поиск трех лучших спортсменов и за $O(N)$. 

Сложность задачи заключается в том, чтобы суметь сравнить результаты двух спортсменов и выбрать лучшего.

Для большей определенности при чтении данных <<проигнорируем>> тех спортсменов, у которых не оказалось зачетных результатов хотя бы по одному виду, то есть тех, у кого либо все попытки в рывке провалены, либо все в толчке, либо в обоих сразу. Если при этом ни одного спортсмена не осталось, выводим <<No results>>. Параллельно
создаем три вспомогательных списка: список данных для вывода (вуз, имя и фамилия), список данных для обработки и список номеров спортсменов. Для того, чтобы упростить обмены, будем переставлять именно номера спортсменов, а не все их данные. В список обрабатываемых данных для каждого спортсмена сохраним: сумму лучших результатов по обоим видам, лучший результат в толчке, номер попытки в толчке, результаты в предыдущей и предпредыдущей попытках (если таковых не окажется, зададим их нулями), а также номер жеребьевки.

В авторском решении данной задачи была использована <<гномья>> сортировка, для сравнения результатов спортсменов была составлена специальная функция \textit{better}. Данная функция скрупулезно реализует описанные в условии задачи правила.

А именно сравнивает сначала суммы лучших результатов по каждому из видов,
при их равенстве~--- лучшие результаты в толчке. При этом выше ранжируется спортсмен, у которого данный результат МЕНЬШЕ (видимо, рывок является более сложным упражнением).

Далее, при равенстве результатов в толчке, рассматриваются номера попыток. Выше ранжируется спортсмен, который взял вес с более ранней попытки. Если и в этом они оказались равны, переходит к сравнению результатов в более ранних попытках (при их наличии). И здесь, если один из спортсменов вообще не взял никакой вес, он ПРОИГРЫВАЕТ.
Если же все попытки толчка у обоих оказались одинаковы, то их судьбу решает жребий~--- выше ранжируется тот спортсмен, который вышел на помост раньше.

Для интересующихся приведем фрагмент программы на языке Python (сортировка и вывод результатов):
\begin{verbatim}
    #гномья сортировка, m - список номеров спортсменов            
    i=k-1
    while i>0:
        if better(m[i], m[i-1]):
            tmp = m[i]
            m[i]= m[i-1]
            m[i-1] = tmp
            if i <k-1:
                i += 1
        else:
            i -= 1

    for i in range(min(3, k)):
        print(athletes[m[i]])
   
\end{verbatim}

\end{tutorial}
