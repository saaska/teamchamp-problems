\begin{tutorial}{Философский камень}

Задача сводится к нахождению НОК$(a_1,a_2,a_3,...,a_n)$. Использование перебора для нахождения НОК множества чисел не подходит для в этого случая. Для решения задачи воспользуемся теоремой, согласно которой НОК последовательности чисел можно вычислить, рекурсивно вычисляя НОК двух чисел:
\begin{center}
$m_1=$НОК$(a_1, a_2)$, $m_2=$НОК$(m_1,a_3)$, ..., $m_k=$НОК$(m_{k-1},a_n)$
\end{center}
тогда
\begin{center}
$m_k=$НОК$(a_1,a_2,a_3,...,a_n)$.
\end{center}
Для нахождения НОК двух чисел необходимо воспользоваться алгоритмом Евклида.

\end{tutorial}
