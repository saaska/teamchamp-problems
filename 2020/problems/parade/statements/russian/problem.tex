\begin{problem}{Философский камень}{стандартный ввод}{стандартный вывод}{1 секунда}{256 мегабайт}

Никола Фламель в этом году нашел тайную книгу заклинаний знаменитого чернокнижника Италии. Эта книга замечательна тем, что в ней содержится рецепт получения философского камня. Но этот рецепт будет иметь силу, только если он приготовлен во время великого противостояния планет (когда все планеты системы выстраиваются в одну линию).
Если известны периоды обращения планет (в земных днях), а также то, что последнее великое противостояние планет было $K$ дней тому назад, ответьте, когда Никола Фламель сможет получить философский камень.

\InputFile
На вход программе дается последовательность целых чисел. Первая строка содержит числа $N$ и $K$~--- количество планет и дней, прошедших с последнего великого противостояния планет ($0 < N \leq 20$, $1\leq K \leq 10^{3}$). На следующей строке заданы разделенные пробелом периоды обращения планет ($1\leq A_i < 10^{2}$, $i=1,\ldots, N$) в днях.

\OutputFile
Ваша программа должна вывести одно единственное целое число~--- наименьшее количество дней, через которые Никола Фламель сможет получить философский камень.

\Example

\begin{example}
\exmpfile{example.01}{example.01.a}%
\end{example}

\end{problem}

