\begin{problem}{Складской учет}{стандартный ввод}{стандартный вывод}{1 секунда}{256 мегабайт}

Уйгуна~--- молодая начинающая предпринимательца, которая недавно открыла магазин. Каждую неделю она должна проводить аудит остатков по товарам и составлять список товаров на пополнение. Если остаток товара в магазине меньше $A$ единиц, то он включается в список на пополнение остатков. Помогите Уйгуне автоматизировать данную операцию, разработайте программу для составления списка товаров.

\InputFile
На первой строке заданы два целых числа $A$ и $N$, минимальный остаток и количество видов товаров в магазине ($0 < A, N \leq 10^{5}$).
В каждой из следующих $N$ строк заданы: разделенные пробелом наименование товара и его остаток на данный момент $a_i$ ($0<a_i \leq 10^{10}$).

\OutputFile
Ваша программа должна вывести список наименований товаров, остатки по которым необходимо пополнить на этой неделе, каждое с новой строки.

\Example

\begin{example}
\exmpfile{example.01}{example.01.a}%
\end{example}

\end{problem}

