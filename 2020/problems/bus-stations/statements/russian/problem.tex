\begin{problem}{Автобусные остановки}{стандартный ввод}{стандартный вывод}{3 секунды}{256 мегабайт}

Один недовольный житель города N опубликовал  в Инстаграме стори о том, как он ежедневно добирается до работы на автобусе, делая по две пересадки. Депутаты специальной комиссии городской думы, после тщательного разбирательства, пришли к выводу, что схема движения городских автобусов неэффективна.

Чтобы решить эту проблему, мэр постановил создать новый автобусный маршрут, который должен будет связать самые неудачные остановки. В своем Твиттере он объявил конкурс, победителем которого станет человек, нашедший наибольшее количество пар неудачно расположенных остановок.

Чтобы разъяснить, что же на самом деле имел в виду мэр, в официальной группе мэрии города~N ВКонтакте был опубликовано следующее коммюнике:
\begin{quote}
\textit{Расстоянием} между двумя остановками будем называть минимальное возможное
количество межостановочных интервалов, которое необходимо проехать по пути от одной из них до другой.\\
Пару остановок назовем \textit{неудачной}, если невозможно найти другую пару остановок с большим расстоянием, чем у неё.\\
Победителем конкурса становится участник, который быстрее остальных сможет найти все пары неудачных остановок.
\end{quote}
Чтобы помочь мэрии быстро оценить все приходящие решения, нужно написать программу, которая определит количество пар неудачных остановок и расстояние между ними.


\InputFile
В первой строке входных данных дается одно целое положительное число $N$~--- количество автобусных маршрутов ($1 \leq N \leq 10^2$).

Во второй строке входных данных дается $N$ целых положительных чисел $K_i$~--- количество остановок на каждом из маршрутов ($1 \leq K_i \leq 10^4$).

В последующих $N$ строках задано по $K_i$ целых положительных чисел $v_{ij}$~--- номера автобусных остановок, составляющих $i$-й маршрут ($1 \leq v_{ij} \leq 10^4$).


\OutputFile
В качестве результата ваша программа должна вывести два целых положительных числа, разделенных пробелом --- количество пар неудачных остановок и расстояние между ними.

\Example

\begin{example}
\exmpfile{example.01}{example.01.a}%
\end{example}

\end{problem}

