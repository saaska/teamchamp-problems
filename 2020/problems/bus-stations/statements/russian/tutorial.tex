\begin{tutorial}{Автобусные остановки}

Задача основана на поиске кратчайшего пути в графе. Ограничения на входные данные позволяют использовать стандартные алгоритмы: поиск в ширину, алгоритм Дейкстры или алгоритм Флойда-Уоршелла.

В простейшем случае, после чтения входных данных следует вычислить все пары расстояний между остановками. Затем определить максимальное расстояние и соответствующее количество вершин графа.

Задача также предоставляет широкое поле для оптимизации. Трудоемкость алгоритма можно значительно снизить следующими способами:
\begin{itemize}
\item Рассматривать только конечные остановки автобусных маршрутов.
\item Изначальный невзвешенный граф можно превратить во взвешенный, приписав ребрам вес равный 1. Затем, можно объединить промежуточные остановки, сложив веса объединяемых ребер.
\end{itemize}

\end{tutorial}
