\begin{tutorial}{RAID 5}

Для упрощения решения создадим алфавит --- одномерный массив, который будет содержать все возможные символы из сообщения. В дальнейшем, чтобы определить код символа, будем искать его в алфавите. Код символа будет совпадать с его индексом (положением).

Организуем цикл который обходит символы в блоках. Для каждого символа суммируем коды из всех блоков. Затем, используя операцию взятия остатка от деления можно получить код соответствующего символа из недостающего блока:
$$
r_i = (64 - s_i) \% 64
$$
где $s_i$ --- сумма кодов $i$-го символа из всех блоков, $r_i$ --- искомый код символа из недостающего блока, и $\%$ --- оператор остатка от деления. 

На языке Python после ввода данных решение может иметь вид:

\begin{verbatim}
alphabet = '0123456789'
alphabet += 'abcdefghijklmnopqrstuvwxyz'
alphabet += 'ABCDEFGHIJKLMNOPQRSTUVWXYZ'
alphabet += '_.'

answer = ''
for i in range(L):
    code = 0
    for j in range(K):
        block = blocks[j]
        symbol = block[i]
        code += alphabet.find(symbol)
    answer += alphabet[(64 - code) % 64]

print(answer)
\end{verbatim}

\end{tutorial}
