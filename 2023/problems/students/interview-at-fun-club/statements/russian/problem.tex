\begin{problem}[(Парников В.\,В.)]{Собеседование в фан-клуб \label{fanclub}}{стандартный ввод}{стандартный вывод}{1 секунда}{256 мегабайт}

Школьник Петя захотел вступить в фан-клуб Леонида Скорнякова. Из-за обилия желающих вступить в него, ему нужно пройти собеседование. На нём ему дали 6 чисел, являющихся слагаемыми одночленами выражения, получающегося при возведении в квадрат суммы трех чисел, в~случайном порядке:
$$(a + b + c)^2 = a^2 + 2 a b + b^2 + 2 b c + c^2 + 2 a c.$$

Пете нужно найти значение $a+b+c$, но он никак не может решить задачу. Петя очень хочет в~фан-клуб и просит вас помочь ему.

\InputFile
В первой строке через пробел записаны 6 целых чисел $x_1, x_2, ..., x_6$ ($1 \leq x_i \leq$ $\leq 10^8$)~--- слагаемые многочлена в~случайном порядке.

\OutputFile
Выведите одно число~--- искомое значение $a + b + c$.

\Example

\begin{example}
\exmpfile{example.01}{example.01.a}%
\end{example}

\Note
Гарантируется, что $a, b, c$~--- целые положительные числа.

\end{problem}

