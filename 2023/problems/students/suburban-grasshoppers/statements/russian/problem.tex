\begin{problem}[(Зыков Т.\,А.)]{Электричка кузнечиков }{стандартный ввод}{стандартный вывод}{1 секунда}{256 мегабайт}

В городе кузнечиков ходили электрички, состоящие из единственного вагона с единственным рядом из $n$ мест. На станцию приехала электричка, в которой все места были свободны. Ждали её $m$ кузнечиков, каждый из которых имеет длину прыжка $a_{i}$, $i=1, \dots, m$. Они по очереди заходят в вагон. Когда заходит $i$-й кузнечик, он сначала проверяет, свободно ли 1-е место. Если там уже кто-то сидит, то кузнечик прыгает на $a_{i}$ мест и проверяет $1 + a_{i}$ место. Если оно опять же занято, то он прыгает дальше, а иначе садится. И так далее заходят все кузнечики по очереди. Если кузнечик не нашел себе место и не может дальше прыгать, то он выходит из электрички. 

Вас попросили оптимизировать движение электричек~--- можно ли заводить кузнечиков так, чтобы уместить в электричке их всех? 

Известно, что все $a_{i}$ --- степени двоек, то есть их можно представить в виде $a_{i} = 2^k$, где $k$~--- неотрицательное целое число.

\InputFile
В первой строке через пробел заданы натуральные числа $n$ и $m$~--- количество мест и кузнечиков соответственно ($1 \leq n \leq 10^4$, $1 \leq m \leq 10^3$).

Во второй строке через пробел заданы длины прыжков кузнечиков $a_{1}, a_{2}, \ldots, a_{m}$ ($1 \leq a_{i} \leq 2^{\log{n} + 1}$, $a_{i} = 2^k$).


\OutputFile
Выведите {\tt YES}, если существует порядок входа кузнечиков, чтобы все поместились, иначе выведите {\tt NO}.
Если существует такой порядок входа, то дополнительно во второй строке нужно вывести индексы кузнечиков через пробел в порядке их входа.

\Examples

\begin{example}
\exmpfile{example.01}{example.01.a}%
\exmpfile{example.02}{example.02.a}%
\end{example}

\end{problem}

