\begin{problem}[(Парников В.\,В.)]{Эстафета с гирями }{стандартный ввод}{стандартный вывод}{1 секунда}{256 мегабайт}

На уроке физкультуры учитель решил провести необычную эстафету с~передачей гирь.

Он выставил в ряд $n$ школьников. Начиная с первого в ряду, они по очереди пробегают круг в зале с гирей и передают её следующему в ряду ($i$-й в ряду передаёт $i+1$-му).

Известно, что $i$-й школьник может поднять максимум $a_i$ килограмм. Если школьник не может поднять гирю, то на нём её передача эстафеты останавливается. Также эстафета заканчивается, если все пробежали по одному кругу. 

Всего прошло $m$ эстафет с гирями весом $b_j$ килограмм.

Помогите учителю посчитать количество кругов в зале, которые пробежал каждый школьник за все эстафеты, чтобы выставить им оценки.

\InputFile
В первой строке записаны два целых числа $n$ и $m$ ($1 \leq n \leq 10^6, 1 \leq m \leq 10^5$) --- количество школьников и количество эстафет соответственно.

Во второй строке записаны $n$ целых чисел $a_1, a_2, ..., a_n$ ($1 \leq a_i \leq 10^9$)~--- максимальный вес, который может поднять каждый $i$-й школьник.

В третьей строке записаны $m$ целых чисел $b_1, b_2, ..., b_m$ ($1 \leq b_j \leq 10^9$) --- веса гирь для каждой $j$-й эстафеты.

Все числа разделены между собой одним пробелом.

\OutputFile
Выведите $n$ целых чисел --- количество кругов, которое пробежал каждый школьник в порядке их расположения в ряду.

\Example

\begin{example}
\exmpfile{example.01}{example.01.a}%
\end{example}

\end{problem}

