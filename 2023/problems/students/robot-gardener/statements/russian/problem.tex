\begin{problem}[(Парников В.\,В.)]{Робот-садовод\label{robotgardener}}{стандартный ввод}{стандартный вывод}{1 секунда}{256 мегабайт}

На грядке с $n$ бамбуками решили автоматизировать процесс собирания урожая. Роботу дали следующую программу:
\begin{enumerate}
  \item Если длина бамбука кратна двум: срубить половину растения за единицу времени.
  \item Если больше не осталось бамбуков с чётной длиной, завершить программу. Иначе перейти к~другому растению и повторить пункты 1 и 2.
\end{enumerate}

Робот, как и любое другое автоматическое устройство, хранит и обрабатывает все числа в~двоичном виде.

Робота запустили у первого растения. Посчитайте время, через которое робот завершит свою программу.



\InputFile
Первая стройка содержит одно целое число $n$ ($1 \leq n \leq 10^3$)~--- количество бамбуков на грядке.

В следующих $n$ строках задано по одному числу $a_i$ ($2 \leq a_i < 2^{1000}$) в~двоичной системе счисления без ведущих нулей~--- длина $i$-го бамбука, $i = 1, \dots, n$.

\OutputFile
Выведите одно число~--- искомое время.

\Example

\begin{example}
\exmpfile{example.01}{example.01.a}%
\end{example}

\Note
Время тратится только на рубку бамбука. На все остальные операции, включая переход между растениями, время не тратится.

\end{problem}

