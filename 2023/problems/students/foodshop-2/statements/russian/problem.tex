\begin{problem}[(Булатов В.\,А.)]{Продуктовый магазин }{стандартный ввод}{стандартный вывод}{1 секунда}{256 мегабайт}

Предприниматель готовится к открытию своего нового магазина. Он уже подготовил стеллажи и закупил необходимый ассортимент товара. Осталось только расставить продукты по полкам. Он просит вас помочь ему сделать эффективную расстановку товаров.

Стеллаж представляет собой длинную полку длиной $L$ дм. Владелец подготовил $N$ различных товаров, которые можно разместить на этом стеллаже. У каждого товара есть определенная длина, которую он займет на полке~--- $x_i$ дм.

Необходимо выбрать множество различных товаров так, чтобы суммарная длина всех выбранных товаров на полке была равна $L$.

\InputFile
В первой строке через пробел записаны 2 целых числа $N$, $L$ ($1 \leq N \leq 1000$, $1 \leq L \leq 1000$).

Во второй строке через пробел заданы $N$ целых чисел $x_1, x_2, \ldots, x_N$ ($1 \leq $\linebreak $\leq x_i \leq 1000$).

\OutputFile
Если подходящая расстановка существует, вывести \texttt{YES}, иначе вывести \texttt{NO}.

\Example

\begin{example}
\exmpfile{example.01}{example.01.a}%
\end{example}

\Note
В примере правильным решением будет, например, вариант с выбором первого и последнего товара.

\end{problem}

