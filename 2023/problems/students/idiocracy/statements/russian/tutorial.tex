\begin{tutorial}{Идиократия}

Пусть мы <<впихнули>> некий $k$-угольник в $n$-угольник. Построим окружность около $n$-угольника. Поскольку мы имеем дело с правильными $n$- и~$k$-угольниками, заметим, что между любыми двумя соседними вершинами $n$-угольника $360 / n$ градусов по дуге, а между любыми двумя соседними вершинами $k$-угольника~--- $360 / k$ градусов. $k$-угольник у нас построен на вершинах $n$-угольника, соответственно, между каждыми двумя соседними вершинами $k$-угольника должно быть одинаковое кол-во вершин $n$-угольника. Выходит, что мы могли <<впихнуть>> $k$-угольник только тогда, когда $n$ кратно $k$. То есть каждая вершина $k$-угольника находится через $n / k$ вершин $n$-угольника.

Находим и выводим максимальный делитель числа $n$, отличный от 1 и $n$. Если число простое, то выводим $-1$.

\end{tutorial}
