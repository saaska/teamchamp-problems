\begin{problem}[(Мекумянов С.\,Л., Ле\-верь\-ев~В.\,С.)]{Закраска прямоугольника}{стандартный ввод}{стандартный вывод}{1 секунда}{256 мегабайт}

Семён~--- молодой и талантливый программист. Он увлекается абстрактной живописью и робототехникой. Однажды ему приснился сон, в~котором робот пытался написать копии нескольких известных картин. Однако в~силу ограниченности конструкции роботу это не всегда удавалось.

Оригинал картины во сне представлял собой черно-белый прямоугольник, состоящий из $N \times M$ клеток одинакового размера. Для простоты их можно обозначить символами <<\texttt{.}>> и <<\texttt{\#}>>~--- непокрашенные и~покрашенные клетки соответственно.

Конструкция робота во сне была такова, что он мог держать в манипуляторе только прямоугольную кисть со сторонами $1 \times K$. Кисть могла перемещаться над полотном картины в любом направлении по горизонтали или вертикали. Также робот мог поворачивать её на любой угол кратный 90\textdegree, пока кисть не прижата к полотну. Однако он не мог выводить часть кисти за пределы полотна картины, так как в~этом случае можно было закрасить раму.

Когда Семён проснулся, ему в~голову пришла программа, которая весьма эффективно определяет, может ли описанный робот нарисовать определённую картину или нет. А сможете ли вы написать подобную программу?

\InputFile
В~первой строке заданы разделенные пробелами целые положительные числа $N, M, K$ $(1 \leq N, M, K \leq 2000)$.

В~последующих $N$ строках записаны по $M$ символов <<\texttt{\#}>> или <<\texttt{.}>>. Символ <<\texttt{\#}>> означает закрашенный элемент, а <<\texttt{.}>> ---  незакрашенный.

\OutputFile
Ваша программа должна вывести ответ на задачу --- одно слово <<YES>> либо <<NO>> (регистр символов не играет роли).

\newpage

\Examples

\begin{example}
\exmpfile{example.01}{example.01.a}%
\exmpfile{example.02}{example.02.a}%
\end{example}

\end{problem}

