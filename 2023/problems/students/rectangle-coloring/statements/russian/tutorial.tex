\begin{tutorial}{Закраска прямоугольника}
\sloppy

Сначала рассмотрим очевидный вариант решения. Выполним обход \linebreak $N$~строк, а внутри него~--- обход $M$ столбцов. Затем от каждой закрашенной и необработанной ранее клетки запустим поиск соседей по горизонтали и по вертикали. Если количество соседей и~по горизонтали, и~по вертикали меньше $K$, то мы нашли клетку, которую невозможно прокрасить.

Данное решение верное, но его трудоемкость~--- $O(N \cdot M \cdot K)$, что при максимальных значениях входных данных приведет к превышению лимита времени. Проблема заключается в последнем цикле, который нужен для поиска количества соседей по горизонтали и вертикали.

Давайте попробуем избавиться от него. Передварительно создадим два вспомогательных массива $H$ и $V$ такого же размера, как и полотно картины и заполним их нулями. Снова выполним обход $N$ строк и~$M$~столбцов, при этом обозначим текущую строку~--- $i$, а столбец~--- $j$. Однако в этот раз будем отмечать закрашенные ячейки числами во вспомогательных массивах $H$ и $V$. Найдя закрашенную ячейку, прочитаем числа, соответствующие уже пройденным ячейкам $H(i, j-1)$, $V(i-1, j)$ (если они есть), увеличим их на 1 и запишем в соответствующие ячейки массивов: $H(i, j) = H(i, j-1) + 1$ и $V(i, j) = V(i, j-1) + 1$.

Теперь организуем обход ячеек картины в обратном порядке: снизу вверх и справа налево. Встретив закрашенную ячейку, сравним значения из вспомогательных массивов $H(i,j)$ и $H(i,j+1)$, а также $V(i,j)$ и~$V(i+1,j)$ (если соответствующие значения существуют). Запишем\linebreak в~$H(i,j)$ и $V(i,j)$ максимальные значения. Таким образом, получится, что мы определили длины всех горизонтальных и вертикальных мазков.

В случае, если найдется закрашенная ячейка, для которой одновременно $H(i,j) < K$ и $V(i,j)  < K$, то это будет означать, что мы нашли ячейку, которую робот не может прокрасить и нужно вывести ответ \texttt{NO}. Если таких ячеек нет, то ответ~--- \texttt{YES}.

Новый вариант решения выполняет несколько простых обходов с~трудоемкостью $O(N \cdot M)$, что позволит уложиться в лимит времени.

\end{tutorial}
