\begin{problem}[(Антонов Ю.\,С.)]{Рыбаки}{стандартный ввод}{стандартный вывод}{1 секунда}{256 мегабайт}

Том Сойер и Гекльберри Финн отправились на рыбалку. Оба они поймали по $n$ рыб. Финн предложил взвесить уловы каждого рыбака, чтобы выяснить, кто из них лучший рыбак. Но Сойер не согласился (была, видимо, причина). Он предложил сложить оба улова и отсортировать по весу в~убывающем порядке. Затем сложить индексы рыб своего улова. Чья сумма индексов окажется меньше, тот и победитель. При этом те рыбы, вес которых одинаков у Финна и Сойера, не учитываются. 

Надо написать программу, определяющую победителя по Сойеру. Если победитель не определяется, то вывести на печать $-1$.




\InputFile
В первой строке задано натуральное число $n$ ($n<1000$). Во второй и~третьей строках через пробел записаны по  $n$ натуральных чисел. Каждое число означает вес рыбы (каждая рыба весит  меньше $1000$). Вторая строка~--- это веса рыб, пойманных Сойером, а третья строка, соответственно, ~--- веса рыб, пойманных Финном.


\OutputFile
Вывести либо $-1$, если не удалось выявить победителя, либо два числа. Первое число~--- сумма индексов рыб Сойера, второе число~--- сумма индексов рыб Финна.


\Examples

\begin{example}
\exmpfile{example.01}{example.01.a}%
\exmpfile{example.02}{example.02.a}%
\end{example}

\end{problem}

