\begin{tutorial}{Рыбаки}

Вначале нужно убрать одинаковые значения в массивах весов рыб у~Сойера и Финна. Для этого можно в двойном цикле по массивам Финна и~Сойера совпадающие значения $A[i]$ и $C[j]$ заменить нулями, затем удалить нули из массивов $A$ и $C$.  

Далее заведем два новых массива $A1$ и $C1$. В массив $C1$ поместим по порядку числа от 1 до $k$, а в~массив  $A1$~--- числа от 1001 до $1001+k$, где $k$~--- это количество оставшихся элементов в~массивах $A$ и $C$. Добавляем к массиву $C$ элементы массива $A$,  а к массиву $C1$~--- элементы массива $A1$. Сортируем массив $C$ по убыванию, при этом в массиве $C1$ делаем такие же обмены, как и в массиве $C$. После этого в преобразованном массиве $C1$ суммируем по отдельности индексы чисел, меньших 1000, и индексы чисел, больших 1000. В случае, если суммы равны, выводим $-1$. Иначе выводим вначале сумму индексов весов рыб Сойера и через пробел аналогичную сумму Финна.



\end{tutorial}
