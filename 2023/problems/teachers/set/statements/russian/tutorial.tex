\begin{tutorial}{Множество чисел}

Число $2023=17 \cdot 17 \cdot 7$. Из массива, где 2021 единица, одна цифра $a$ и одна цифра $b$, будем составлять разные массивы длины 7, 17, 119, 289. Предполагаем вначале, что они состоят из  одних единиц, затем, что кроме единиц они содержат одну цифру $a$, затем, что кроме единиц они содержат одну цифру $b$, и, наконец, что кроме единиц они содержат одну цифру $a$ и одну цифру $b$. В каждом из этих массивов,  кроме массивов, состоящих из одних единиц, цифры $a$ и $b$ могут стоять в любом месте. Мы должны из этих массивов составлять числа и определять, сколько из них делится на 2023. 

Рассмотрим алгоритм перебора чисел длины $n$, составленных из массива  единиц и одной цифры~$a$.

В массиве из одних единиц организуем цикл по $i$ от 1 до $n$, $i$-ю цифру заменяем на цифру $a$, составляем из массива число, проверяем делимость на 2023. Если делится, к счетчику делимости прибавляем единицу, снова превращаем число в массив, заменяем  $i$-ю цифру на цифру $1$, увеличиваем $i$. 

Те же самые действия повторяем для цифры $b$.

Рассмотрим теперь алгоритм перебора чисел длины $n$, составленных из массива единиц, одной цифры $a$ и одной цифры $b$. 

Организуем цикл по $i$ от 1 до $n$. Заменяем $i$-ю цифру на цифру $a$, организуем цикл по $j$ от 1 до $n$. Если $j$-я цифра~--- единица, заменяем эту цифру на цифру $b$, составляем из массива число, проверяем делимость на 2023. Если число делится на 2023, к~счетчику делимости прибавляем 1. Снова превращаем число в массив, заменяем $j$-й элемент на цифру $1$, увеличиваем $j$. По окончании цикла по $j$ заменяем $i$-ю цифру на цифру $1$, увеличиваем $i$. 

После окончания всех этих переборов к счетчику делимости добавим по единице, если числа, составленные соответственно из массивов длины 7,  17, 119, 289 и состоящие сплошь из единиц, делятся на 2023.    

\end{tutorial}
