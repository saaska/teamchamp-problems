\begin{tutorial}{Продуктовый магазин}

Данная задача является классической задачей на динамическое программирование~--- задачей <<о рюкзаке>> без стоимости.

Обозначим как $P_s$ множество возможных длин, которые можно получить, используя товары из множества $s$. Очевидно, что для пустого множества товаров ($\varnothing$) можно получить только длину 0: $$P_\varnothing = \{0\}.$$
Для множества из одного товара ${y_1}$: $$P_{\{y_1\}} = \{0, y_1\}.$$
Интуитивно можно вывести следующую закономерность: $$P_{\{y_1, y_2, \ldots y_n\}} = P_{\{y_1, y_2, \ldots y_{n-1}\}} \cup \left\{  t + y_n \Big| t \in P_{\{y_1, y_2, \ldots y_{n-1}\}} \right\}.$$

Последовательно вычислив $P_{\varnothing}$, $P_{\{x_1\}}$, $P_{\{x_1, x_2\}}$, $\ldots$, $P_{\{x_1, \ldots, x_N\}}$, можно определить ответ: <<YES>>, если $L \in P_{\{x_1, \ldots, x_N\}}$, иначе <<NO>>. Для оптимизации работы алгоритма следует учитывать только значения, не превышающие $L$.

\end{tutorial}
