\begin{problem}{Простая игра}{стандартный ввод}{стандартный вывод}{1 секунда}{256 мегабайт}

Учёный по имени Джеймс играет с демоном в игру, цель которой проста --- нужно угадать загаданное демоном число $N$ ($1 < N < 10^4$).

Игрок делает догадку, а демон может ответить:

\begin{itemize}\setlength\itemsep{0pt}
\item OK --- если игрок угадал число;
\item YES --- если загаданное число делится на догадку;
\item NO --- если загаданное число НЕ делится на догадку.
\end{itemize}

Джеймс уже давно играет с демоном и нашел хорошую стратегию, которая позволяет угадать загаданное число за минимальное количество попыток.

Сможете ли вы написать программу, которая угадывает число не хуже него?

\Interaction
Программа должна использовать стандартный вывод, чтобы осуществлять запросы.

Каждый запрос --- это целое положительное число $q$ ($1 < q < 10^4$). После вывода числа нужно вывести символ перевода строки.

Программа должна использовать стандартный ввод, чтобы читать ответы на запросы.

Программа будет получать по одному слову в каждой строке. Если пришло слово <<OK>>, то ваша программа должна завершить свою работу.

Тестирующая система даст прочитать очередную строку только после записи в стандартный вывод нового запроса.

\Example

\begin{example}
\exmpfile{example.01}{example.01.a}%
\end{example}

\Note
Ваша программа получит вердикт <<WA>> если Джеймс сможет угадать число за меньшее число попыток чем ваша программа.

В примере демон загадывает число <<6>>.

Джеймс спрашивает 2? Демон отвечает - <<YES>>.

Джеймс спрашивает 3? Демон отвечает - <<YES>>.

Джеймс спрашивает 5? Демон отвечает - <<NO>>.

Джеймс спрашивает 6? Демон отвечает - <<ОК>>.

Число в ответе на пример --- это количество выполненых попыток.

\end{problem}

