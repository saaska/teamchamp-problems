\begin{tutorial}{Эстафета с гирями}

Отсортируем гири по убыванию массы. Пройдемся по школьникам в их порядке в ряду, храня указатель на максимальный вес, который проходит до текущего школьника. Понятно, что количество кругов, которые пробежит школьник, равно количеству гирь, вес которых не больше веса гири, на которую установлен указатель (гири, вес которых больше, просто не дойдут до этого школьника). Это количество легко находится как общее количество гирь минус индекс указателя.

\end{tutorial}
