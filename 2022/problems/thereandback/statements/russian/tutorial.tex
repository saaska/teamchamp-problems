\begin{tutorial}{Успеть к финалу}

Время, необходимое для того, чтобы сплавать за отцом и вернуться домой, можно вычислить по формуле:
$$T=\frac{L}{v_b-v_r} + \frac{L}{v_b+v_r}.$$
Но в этом случае время будет в часах, а время до финала задано в минутах. Поэтому $T$ нужно перевести в минуты~--- умножить его на $60$, тогда получим:
$$T=60 \cdot \frac{L}{v_b-v_r} + 60 \cdot \frac{L}{v_b+v_r}.$$
Быhый успеет, если $a\leq T$, но по условию задачи два числа с плавающей точкой равны между собой, если модуль разности этих двух чисел не превышает заданного $\epsilon$. Тогда условие того, что Быhый успеет к началу матча будет выглядеть следующим образом:

\center{$T<a$ or $abs(T-a) \leq \epsilon$.}

\end{tutorial}
