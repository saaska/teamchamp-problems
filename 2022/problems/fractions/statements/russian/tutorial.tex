\begin{tutorial}{Дроби}

Основная проблема данной задачи в том, что дроби могут быть приведены в виде сократимой дроби. Для решения этой проблемы нужно уметь приводить их к нормальной несократимой форме (числитель и знаменатель взаимно простые и знаменатель больше нуля). Для приведения дробей к нормальной форме нужно поделить числитель и знаменатель на их наибольший общий делитель.

Для вычисления суммы/разности дробей нужно привести их к общему знаменателю и сложить/вычесть числители. Полученную дробь также нужно нормализовать.

Перейдем к решению самой задачи. Заведем словарь~--- ключом будет выступать нормализованная дробь, а значением будет представление дроби во входных данных. Проходим по всем карточкам с дробями. Для каждой дроби выполняем следующие действия:
\begin {itemize}
\item вычисляем разность требуемой дроби $\frac{a}{b}$ и $\frac{x_i}{y_i}$;
\item проверяем наличие этой разности в словаре, если дробь нашлась~--- ответом являются дроби $\frac{x_i}{y_i}$ и значение в словаре, иначе вычисляем нормализованную дробь и записываем ее в словарь.
\end{itemize}

\end{tutorial}
