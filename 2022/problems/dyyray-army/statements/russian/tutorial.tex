\begin{tutorial}{Армия Дыырая}

Докажем, что мы всегда можем разбить воинов на $\displaystyle \frac{n}{2}$ простых пар. Сделаем это по индукции.

Для $n = 2$ разбиение очевидно и единственно: $1 + 2 = 3$.

Допустим, что разбиение на простые пары существует для всех $n$ от $2$ до $2(k-1)$. Докажем,  что оно существует и для $n=2k$. Приведем такое разбиение.

По теореме \textit{Бертрана~--- Чебышёва} для любого натурального $n \ge 2$ найдётся простое число $p$ в интервале $n < p < 2n$. Значит найдется такое $m$ $(1 \le m < 2k)$, что $2k + m$~--- простое $(2k < 2k + m < 4k)$, причем $m$~--- нечетное. По предположению индукции разбиение чисел от $1$ до $m-1$ существует. Числа от $m$ до $2k$ будем разбивать следующим образом: 
$$m + 2k,~(m+1) + (2k-1),~...,~\frac{2k + m-1}{2} + \frac{2k + m+1}{2}.$$
Значит, для любого четного $n$, числа от $1$ до $n$ можно разбить на $\displaystyle \frac{n}{2}$ простых пар.

При решении задачи достаточно найти наибольшее $m$, такое, что $m + n$ простое, вывести разбиение чисел от $m$ до $n$, приведенное в доказательстве, и дальше решать задачу для $n = m-1$.

Ниже приведено решение задачи на языке \texttt{Python}:
{\small
\begin{verbatim}
def is_prime(x):
    d = 2
    while d*d <= x:
        if x % d == 0:
            return False
        d += 1
    return True

def print_primes(l, r):
    while l < r:
        print(l, r)
        l += 1
        r -= 1

n = int(input())

while n > 0:
    x = n - 1
    while not is_prime(x + n):
        x -= 1
    print_primes(x, n)
    n = x - 1
\end{verbatim}
}

\end{tutorial}
