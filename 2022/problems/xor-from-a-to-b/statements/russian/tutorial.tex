\begin{tutorial}{Новое слово в криптографии}

\newcommand{\XOR}{\mathop{\mathrm{XOR}}}
Решение задачи основывается на следующем факте <<$A \XOR  B \XOR  B = A$>>. Тогда для нахождения$ \XOR$  всех чисел от $A$ до $B$ можно воспользоваться следующей формулой:
$$(A) \XOR  (A+1) \XOR  (A+2) \XOR \ldots\XOR  (B) =$$
$$((0)\XOR (1)\XOR (2)\XOR \ldots\XOR (A-1))\XOR $$
$$((0)\XOR (1)\XOR (2)\XOR \ldots\XOR (B)).$$

$\XOR$  всех чисел от $0$ до $X$, в свою очередь, достаточно легко выводится. Для этого, например, выпишите первые $16$ двоичных чисел и посмотрите, что будет, если применить операцию $\XOR$  для них всех. Легко заметить, что начиная с~$0$, $\XOR$  каждой следующей пары чисел дает двоичную единицу (0 $\XOR$  1 = 1, 10 $\XOR$  11 = 1 и т.д.). А далее каждые две единицы <<нейтрализуют>> друг друга. Поэтому рассматриваем остаток от деления $X$ на 4:
\begin{itemize}
  \item если остаток равен 0, то $\XOR$  всех чисел до него самого равен 0, и~ответом является число $X$;
  \item если остаток равен 1, то $\XOR$  всех чисел вместе с ним равен 1;
  \item если остаток равен 2, то $\XOR$  всех чисел до него самого равен 1, и~ответом является сумма 1 и числа $X$;
  \item если остаток равен 3, то $\XOR$  всех чисел вместе с ним равен 0.
\end{itemize}

Определив функцию нахождения $\XOR$  всех чисел от $0$ до $X$, далее остается только получить результат по формуле $\XOR(\max(A-1,0)) \mathop{\mbox{\textasciicircum}} \XOR(B)$, где \textasciicircum~--- операция $\XOR$, определенная во многих языках программирования. 

\end{tutorial}
