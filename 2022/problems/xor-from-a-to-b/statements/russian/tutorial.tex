\begin{tutorial}{Новое слово в криптографии}

Решение задачи основывается на следующем факте <<$A$ \textbf{XOR} $B$ \textbf{XOR} $B$ = $A$>>. Тогда для нахождения \textbf{XOR} всех чисел от $A$ до $B$ можно воспользоваться следующей формулой:
$$(A) \textbf{XOR} (A+1) \textbf{XOR} (A+2) \textbf{XOR}\ldots\textbf{XOR} (B) =$$
$$((0)\textbf{XOR}(1)\textbf{XOR}(2)\textbf{XOR}\ldots\textbf{XOR}(A-1))\textbf{XOR}$$
$$((0)\textbf{XOR}(1)\textbf{XOR}(2)\textbf{XOR}\ldots\textbf{XOR}(B)).$$

\textbf{XOR} всех чисел от $0$ до $X$, в свою очередь, достаточно легко выводится. Для этого, например, выпишите первые $16$ двоичных чисел и посмотрите, что будет, если применить операцию \textbf{XOR} для них всех. Легко заметить, что начиная с~$0$, \textbf{XOR} каждой следующей пары чисел дает двоичную единицу (0 \textbf{XOR} 1 = 1, 10 \textbf{XOR} 11 = 1 и т.д.). А далее каждые две единицы <<нейтрализуют>> друг друга. Поэтому рассматриваем остаток от деления $X$ на 4:
\begin{itemize}
  \item если остаток равен 0, то \textbf{XOR} всех чисел до него самого равен 0, и ответом является число $X$;
  \item если остаток равен 1, то \textbf{XOR} всех чисел вместе с ним равен 1;
  \item если остаток равен 2, то \textbf{XOR} всех чисел до него самого равен 1, и ответом является сумма 1 и числа $X$;
  \item если остаток равен 3, то \textbf{XOR} всех чисел вместе с ним равен 0.
\end{itemize}

Определив функцию нахождения \textbf{XOR} всех чисел от $0$ до $X$, далее остается только получить результат по формуле $XOR(\max(A-1,0)) \textasciicircum XOR(B)$, где $\textasciicircum$~--- операция \textbf{XOR}, определенная во многих языках программирования. 

\end{tutorial}
