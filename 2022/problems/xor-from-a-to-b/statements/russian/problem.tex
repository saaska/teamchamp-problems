\begin{problem}{Новое слово в криптографии}{стандартный ввод}{стандартный вывод}{1 секунда}{256 мегабайт}

Начинающий программист Даринка с недавних пор увлеклась криптографией и понимает, что в~основе шифрования сигналов лежат побитовые операции. 

Для решения новой очень важной задачи шифрования ей нужно найти результат \textbf{побитового исключающего или} всех чисел от $A$ до $B$, но у неё это совсем не получается.

Может, вы сможете ей помочь?

\InputFile
В единственной строке заданы два целых числа $A$ и $B$ ($0 \leq A \leq B \leq 10^{18}$).

\OutputFile
Вывести одно целое число~--- ответ на задачу.

\Examples

\begin{example}
\exmpfile{example.01}{example.01.a}%
\exmpfile{example.02}{example.02.a}%
\end{example}

\Note
Исключающее ИЛИ (XOR) устанавливает значение бита результата в 1, если значения в~соответствующих битах исходных переменных различны:\\
1001 XOR 0011 = 1010.

\end{problem}

