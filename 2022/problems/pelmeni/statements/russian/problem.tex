\begin{problem}{Рецепт пельменей}{стандартный ввод}{стандартный вывод}{1 секунда}{256 мегабайт}

Успешный спортивный программист Айтал не смог пройти собеседование в IT-компанию \textit{<<Тындекс>>}. Чтобы хоть как-то платить за подписку на стриминговый сервис \textit{<<Кекфликс>>}, он устроился на работу кассиром в магазин \textit{<<Четвёрочка>>}.

У \textit{<<Кекфликс>>} очень дорогая подписка, а зарплаты кассира не хватает. Поэтому Айтал решил вечерами подрабатывать таксистом в компании \textit{<<outDriver>>}. Как-то раз Айтал взял заказ у CEO \textit{<<outDriver>>}~--- Айсена Момского. Айтал, не зная кого везет, начал рьяно ругать приложение. Айсен Момский и так был не в духе, а слова \textit{<<А вот у Тындекса приложение куда лучше>>} окончательно лишили Айтала работы.

Расстроившись, Айтал пошёл домой смотреть сериалы. По пути домой он решил рискнуть и купить пельмени у конкурента Четвёрочки~--- магазина \textit{<<Магнат>>}. Но разведывательное управление Четвёрочки не дремлет~--- его тут же уволили. Так Айтал стал безработным. 

Вскоре закончилась подписка на \textit{<<Кекфликс>>}, а от Айтала ушла девушка. Лишившись работы, девушки и сериалов, Айтал начал писать рэп. 

Жюри не хочет, чтобы спортивные программисты становилось рэперами, поэтому дает вам для подготовки задачу, которую Айтал не смог решить на собеседовании.

Дан массив длины $n$, пронумерованный от $1$ до $n$, изначально состоящий из нулей, и $m$ запросов. В каждом запросе дано число $x$. Запросы имеют два типа:
\begin{itemize}
  \item \texttt{Ультралевые}~--- среди всех элементов массива с индексами от $1$ до $x$ выбирается наименьший. Если таких несколько, то выбирается элемент с \textbf{наименьшим} индексом;
  \item \texttt{Ультраправые}~--- среди всех элементов массива с индексами от $x$ до $n$ выбирается наименьший. Если таких несколько, то выбирается элемент с \textbf{наибольшим} индексом.
\end{itemize}
Выбранный элемент увеличивается на один.

Требуется вывести полученный массив.

\InputFile
В первой строке заданы три целых числа $n$ и $m$ $(1 \le n, m \le 2 \cdot 10^5)$~--- длина массива и количество запросов.

В следующих $m$ строках даны индекс $x$ $(1 \le x \le n)$ и код $c$~--- описание запроса. Код равен <<\texttt{l}>>, если запрос ультралевый, и <<\texttt{r}>>, если запрос ультраправый.

\OutputFile
Выведите $n$ целых чисел~--- полученный массив.

\Examples

\begin{example}
\exmpfile{example.01}{example.01.a}%
\exmpfile{example.02}{example.02.a}%
\end{example}

\end{problem}

